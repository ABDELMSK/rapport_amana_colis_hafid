\chapter*{Introduction Générale}
\addcontentsline{toc}{chapter}{Introduction Générale}

L'évolution rapide des technologies de l'information et la transformation numérique des entreprises constituent aujourd'hui des enjeux majeurs pour les organisations, particulièrement dans le secteur logistique et postal. Dans ce contexte, le développement de systèmes informatiques performants, sécurisés et évolutifs représente un défi technique permanent pour les ingénieurs informatiques. Cette problématique se trouve au cœur des préoccupations de Barid Al Maghrib (Poste Maroc), opérateur postal national engagé dans une démarche d'innovation technologique continue.

Dans le cadre de ma formation d'ingénieur en informatique à l'École Marocaine des Sciences de l'Ingénieur, j'ai eu l'opportunité d'effectuer un stage de six semaines au sein de Barid Al Maghrib, du 1er juillet au 8 août 2025. Cette expérience professionnelle s'inscrit parfaitement dans la continuité de mon cursus académique, me permettant d'appliquer concrètement les connaissances théoriques acquises durant ma formation et de développer une vision pratique des enjeux technologiques contemporains.

Le projet qui m'a été confié porte sur le développement et l'optimisation d'un système de gestion des colis postaux, une problématique cruciale pour l'efficacité opérationnelle de l'entreprise. Le système existant présentait plusieurs limitations significatives, notamment en matière de performances lors du traitement de gros volumes de données, de qualité de l'expérience utilisateur, et de sécurisation des accès aux informations sensibles. Ces défis techniques nécessitaient une approche d'ingénieur méthodique, alliant analyse approfondie des besoins, conception d'architectures innovantes, et implémentation de solutions technologiques avancées.

Mes contributions se sont articulées autour de trois axes principaux d'innovation technique. Premièrement, j'ai conçu et développé une architecture de données hybride intégrant MongoDB aux côtés du système MySQL existant, permettant d'optimiser significativement les performances des requêtes de consultation et de recherche. Deuxièmement, j'ai entièrement repensé le système de visualisation cartographique en remplaçant l'affichage basique de frontières GeoJSON par une solution avancée utilisant des tiles cartographiques haute résolution et un système de clustering intelligent. Troisièmement, j'ai implémenté une solution complète de sécurisation basée sur Keycloak, garantissant une authentification robuste et une gestion fine des autorisations utilisateur.

Cette mission m'a permis de mobiliser et d'approfondir mes compétences techniques dans des domaines variés : développement backend avec Spring Boot et Spring Security, développement frontend React, gestion de bases de données relationnelles et NoSQL, intégration de solutions d'authentification enterprise, et optimisation de performances système. Au-delà des aspects purement techniques, ce stage a également enrichi ma compréhension des enjeux métier du secteur postal et logistique, ainsi que ma capacité à concevoir des solutions technologiques répondant aux besoins opérationnels réels d'une grande organisation.

L'objectif de ce rapport est de présenter de manière structurée et détaillée l'ensemble de mes contributions techniques, en mettant l'accent sur la démarche d'ingénieur adoptée pour résoudre les problématiques identifiées. Cette démarche comprend l'analyse approfondie de l'existant, la conception d'architectures adaptées, l'implémentation de solutions innovantes, et l'évaluation rigoureuse des résultats obtenus. Ce document illustre également les difficultés rencontrées et les stratégies mises en œuvre pour les surmonter, témoignant ainsi du processus d'apprentissage et de développement professionnel que représente cette expérience.

Le présent rapport s'articule autour de plusieurs chapitres complémentaires. Nous débuterons par une présentation détaillée de Barid Al Maghrib, de son positionnement stratégique et de ses enjeux technologiques. Nous analyserons ensuite le contexte technique du projet, les problématiques identifiées et les objectifs définis. Les chapitres suivants détailleront les phases d'analyse, de conception, de développement et de mise en œuvre des solutions, avant de présenter les résultats obtenus et les perspectives d'évolution. Nous conclurons par une réflexion sur les apports personnels et professionnels de cette expérience, ainsi que sur les compétences développées dans le cadre de cette mission d'ingénieur.