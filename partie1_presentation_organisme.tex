\chapter{Présentation de l'Organisme d'Accueil}

\section{Barid Al Maghrib : Histoire et Mission}

\subsection{Historique de l'entreprise}

Barid Al Maghrib, connue sous l'appellation commerciale "Poste Maroc", constitue l'opérateur postal national du Royaume du Maroc. Créée en 1913 sous le Protectorat français sous le nom de "Postes, Télégraphes et Téléphones du Maroc" (PTT), l'entreprise a connu une évolution significative tout au long du XXe siècle. Après l'indépendance du Maroc en 1956, elle devient "Office National des Postes et Télécommunications" (ONPT), puis se transforme en société anonyme en 1998 sous la dénomination actuelle de Barid Al Maghrib.

Cette transformation en société anonyme marque un tournant stratégique majeur, permettant à l'entreprise de s'adapter aux exigences d'un marché postal en mutation et de développer une approche commerciale moderne. Aujourd'hui, Barid Al Maghrib s'impose comme un acteur incontournable du secteur postal marocain, avec plus d'un siècle d'expérience et une présence territoriale exceptionnelle sur l'ensemble du territoire national.

L'entreprise a su évoluer avec son époque, passant d'un simple service postal traditionnel à un groupe diversifié offrant une gamme complète de services postaux, financiers et logistiques. Cette évolution témoigne de sa capacité d'adaptation et de son engagement constant envers l'innovation et la modernisation de ses services.

\subsection{Mission et vision stratégique}

La mission de Barid Al Maghrib s'articule autour de quatre axes stratégiques fondamentaux. Premièrement, l'entreprise vise à assurer un service postal universel de qualité, garantissant l'accessibilité des services postaux à tous les citoyens marocains, quel que soit leur lieu de résidence. Cette mission de service public constitue le socle historique de l'entreprise et demeure au cœur de ses préoccupations.

Deuxièmement, Barid Al Maghrib s'engage dans le développement de services financiers inclusifs, contribuant ainsi à l'inclusion financière des populations rurales et urbaines. Cette dimension financière représente un pilier essentiel de la stratégie d'entreprise, permettant d'offrir des solutions bancaires accessibles et adaptées aux besoins locaux.

Troisièmement, l'entreprise développe une expertise logistique avancée, positionnant le Maroc comme un hub logistique régional. Cette ambition s'appuie sur les avantages géographiques du royaume et sur l'expertise opérationnelle développée par Barid Al Maghrib au fil des décennies.

Enfin, la vision stratégique de l'entreprise intègre résolument la transformation numérique comme levier de modernisation et de compétitivité. Cette orientation se traduit par des investissements conséquents dans les technologies de l'information, l'automatisation des processus, et le développement de plateformes digitales innovantes.

\subsection{Positionnement sur le marché postal marocain}

Barid Al Maghrib occupe une position dominante sur le marché postal marocain, avec une part de marché de plus de 80\% dans les services postaux traditionnels. Cette position privilégiée s'appuie sur plusieurs avantages concurrentiels durables : une infrastructure territoriale unique avec plus de 1 800 points de contact répartis sur l'ensemble du territoire, une expertise logistique reconnue, et une marque de confiance établie auprès des consommateurs marocains.

L'entreprise fait face néanmoins à une concurrence croissante, notamment dans le secteur de la logistique e-commerce où de nouveaux acteurs privés, tant nationaux qu'internationaux, développent des solutions alternatives. Cette concurrence stimule l'innovation et pousse Barid Al Maghrib à accélérer sa transformation digitale pour maintenir son leadership.

Le positionnement stratégique de l'entreprise repose également sur sa capacité à exploiter les synergies entre ses différents métiers : postal, financier et logistique. Cette approche intégrée constitue un différenciateur concurrentiel significatif, permettant d'offrir des solutions complètes et adaptées aux besoins évolutifs des clients, qu'ils soient particuliers, entreprises ou administrations publiques.

\section{Organisation et Structure}

\subsection{Organigramme général}

L'organisation de Barid Al Maghrib s'articule autour d'une structure matricielle moderne, combinant une approche fonctionnelle et une organisation par métiers. Au niveau de la gouvernance, l'entreprise est dirigée par un Directeur Général, assisté d'un Comité de Direction composé des directeurs des principales divisions opérationnelles et support.

Cette structure organisationnelle privilégie la transversalité et la collaboration entre les différentes entités, favorisant ainsi l'agilité opérationnelle et la capacité d'innovation. Le Comité de Direction assure la coordination stratégique et veille à l'alignement des objectifs opérationnels avec la vision globale de l'entreprise.

L'organigramme intègre également des structures de gouvernance spécialisées, notamment un Comité d'Audit et des Risques, un Comité Technique, et diverses commissions thématiques chargées de piloter les projets transversaux et les initiatives de transformation.

\subsection{Départements et divisions}

L'organisation opérationnelle de Barid Al Maghrib se structure autour de plusieurs divisions métiers principales. La Division Postale gère l'ensemble des activités traditionnelles de courrier, de colis, et de services aux particuliers. Cette division constitue le cœur historique de l'entreprise et emploie la majorité des effectifs opérationnels.

La Division Financière développe et commercialise l'ensemble des produits et services financiers, incluant les comptes courants postaux, les produits d'épargne, les services de transfert d'argent, et les solutions de paiement électronique. Cette division connaît une croissance soutenue et représente un enjeu stratégique majeur pour l'avenir de l'entreprise.

La Division Logistique se spécialise dans les services de messagerie express, la logistique e-commerce, et les solutions supply chain pour les entreprises. Cette division bénéficie de la croissance du commerce électronique et de l'externalisation logistique par les entreprises marocaines.

Les divisions support incluent la Direction des Systèmes d'Information, la Direction des Ressources Humaines, la Direction Financière et Comptable, et la Direction de la Communication et du Marketing. Ces entités assurent les fonctions transversales nécessaires au bon fonctionnement de l'ensemble de l'organisation.

\subsection{Effectifs et ressources humaines}

Barid Al Maghrib emploie plus de 11 000 collaborateurs répartis sur l'ensemble du territoire national, ce qui en fait l'un des principaux employeurs du secteur des services au Maroc. Cette force de travail se caractérise par sa diversité professionnelle, allant des métiers opérationnels traditionnels (facteurs, guichetiers, agents de tri) aux métiers techniques spécialisés (informaticiens, logisticiens, conseillers financiers).

La politique de gestion des ressources humaines de l'entreprise privilégie le développement des compétences et l'adaptation aux évolutions technologiques. Des programmes de formation continue sont déployés régulièrement pour accompagner la transformation digitale et maintenir l'employabilité des collaborateurs.

L'entreprise accorde également une attention particulière à la promotion de la diversité et de l'égalité professionnelle, avec une représentation équilibrée entre hommes et femmes dans les effectifs et un accès égal aux postes de responsabilité. Cette approche inclusive constitue un atout pour l'attractivité employeur et la performance organisationnelle.

\section{Services et Activités}

\subsection{Services postaux traditionnels}

Les services postaux traditionnels de Barid Al Maghrib couvrent l'ensemble des besoins de communication et d'envoi des particuliers et des entreprises. Le service courrier assure la collecte, le tri, le transport et la distribution du courrier ordinaire et recommandé sur l'ensemble du territoire national et international. Ce service bénéficie d'une organisation logistique éprouvée et d'un réseau de distribution capillaire unique au Maroc.

Le service colis constitue un segment en forte croissance, porté par le développement du e-commerce. Barid Al Maghrib propose une gamme complète de solutions d'expédition, depuis le colis standard jusqu'aux services express avec garantie de délai. L'entreprise investit massivement dans la modernisation de ses centres de tri et dans le développement d'outils de tracking avancés pour améliorer la qualité de service.

Les services aux particuliers incluent également la philatélie, les boîtes postales, les services de changement d'adresse, et diverses prestations administratives. Ces services, bien que représentant un volume d'activité plus réduit, contribuent à maintenir la proximité avec les clients et à préserver le rôle de service public de l'entreprise.

\subsection{Services financiers}

L'activité financière de Barid Al Maghrib s'est considérablement développée au cours des dernières années, positionnant l'entreprise comme un acteur financier de référence au Maroc. Les comptes courants postaux (CCP) constituent le produit phare, avec plus de 2,5 millions de comptes actifs. Ces comptes offrent des services bancaires de base adaptés aux besoins des particuliers et des petites entreprises.

Les produits d'épargne incluent le livret postal, des comptes à terme, et des solutions d'épargne logement. Ces produits bénéficient de taux attractifs et de conditions d'accès simplifiées, contribuant ainsi à l'inclusion financière des populations rurales et des segments de clientèle traditionnellement non bancarisés.

Les services de transfert d'argent représentent un segment stratégique, particulièrement pour les flux financiers entre le Maroc et l'étranger. Barid Al Maghrib a développé des partenariats avec les principaux opérateurs internationaux de transfert d'argent, permettant d'offrir des solutions compétitives et sécurisées.

\subsection{Services logistiques et e-commerce}

Face à l'essor du commerce électronique au Maroc, Barid Al Maghrib a développé une offre logistique spécialisée répondant aux besoins spécifiques des marchands en ligne. La solution "E-commerce" propose un service intégré incluant la collecte, le stockage, la préparation de commandes, l'expédition, et la gestion des retours.

L'entreprise a investi dans des plateformes logistiques modernes équipées de systèmes de gestion d'entrepôt (WMS) et d'outils de tracking en temps réel. Ces investissements permettent d'offrir des niveaux de service compatibles avec les exigences du e-commerce, notamment en matière de délais de livraison et de visibilité sur les envois.

Les services logistiques s'étendent également aux solutions B2B, incluant la logistique contractuelle pour les entreprises, la gestion de stocks déportés, et les services de distribution spécialisée. Cette diversification permet à Barid Al Maghrib de capter une part croissante du marché logistique marocain en croissance.