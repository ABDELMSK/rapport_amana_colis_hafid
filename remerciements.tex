\chapter*{Remerciements}
\addcontentsline{toc}{chapter}{Remerciements}

Au terme de ce stage d'ingénieur de six semaines effectué au sein de Barid Al Maghrib (Poste Maroc), je tiens à exprimer ma profonde gratitude envers toutes les personnes qui ont contribué à la réussite de cette expérience enrichissante.

Je souhaite tout d'abord remercier chaleureusement \textbf{Monsieur Achraf}, mon encadrant industriel, pour son accueil, sa disponibilité et ses précieux conseils tout au long de cette période. Sa expertise technique et sa confiance m'ont permis de mener à bien les missions qui m'ont été confiées, notamment l'optimisation du système de gestion des colis postaux avec l'intégration de MongoDB et l'amélioration des visualisations cartographiques.

Mes sincères remerciements s'adressent également à \textbf{Madame Asmaa El Ouerkhaoui}, mon encadrante pédagogique de l'EMSI, pour son suivi attentif, ses orientations méthodologiques et ses retours constructifs qui ont enrichi ma démarche d'ingénieur tout au long de ce projet.

Je remercie vivement l'ensemble des équipes techniques de Barid Al Maghrib qui m'ont accueilli avec bienveillance et ont facilité mon intégration. Leur collaboration et leur partage d'expérience ont été essentiels pour comprendre les enjeux métier et adapter mes solutions techniques aux besoins réels de l'organisation.

Ma reconnaissance va aussi à la direction de Barid Al Maghrib pour m'avoir offert l'opportunité de réaliser ce stage dans un environnement professionnel stimulant, au cœur des défis de la transformation numérique du secteur postal marocain.

Je tiens à remercier l'École Marocaine des Sciences de l'Ingénieur, pour la formation solide qui m'a permis d'aborder sereinement les aspects techniques de ce projet, notamment la maîtrise des technologies Spring Boot, React.

Enfin, j'exprime ma gratitude envers ma famille et mes proches pour leur soutien indéfectible et leurs encouragements qui ont été une source de motivation constante durant cette période.

Ce stage a représenté une étape cruciale dans mon parcours de formation d'ingénieur, me permettant de confronter mes connaissances théoriques aux réalités du terrain et de développer une vision concrète des enjeux technologiques contemporains dans le secteur logistique.